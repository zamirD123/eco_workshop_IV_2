% Options for packages loaded elsewhere
\PassOptionsToPackage{unicode}{hyperref}
\PassOptionsToPackage{hyphens}{url}
%
\documentclass[
]{article}
\usepackage{amsmath,amssymb}
\usepackage{lmodern}
\usepackage{ifxetex,ifluatex}
\ifnum 0\ifxetex 1\fi\ifluatex 1\fi=0 % if pdftex
  \usepackage[T1]{fontenc}
  \usepackage[utf8]{inputenc}
  \usepackage{textcomp} % provide euro and other symbols
\else % if luatex or xetex
  \usepackage{unicode-math}
  \defaultfontfeatures{Scale=MatchLowercase}
  \defaultfontfeatures[\rmfamily]{Ligatures=TeX,Scale=1}
\fi
% Use upquote if available, for straight quotes in verbatim environments
\IfFileExists{upquote.sty}{\usepackage{upquote}}{}
\IfFileExists{microtype.sty}{% use microtype if available
  \usepackage[]{microtype}
  \UseMicrotypeSet[protrusion]{basicmath} % disable protrusion for tt fonts
}{}
\makeatletter
\@ifundefined{KOMAClassName}{% if non-KOMA class
  \IfFileExists{parskip.sty}{%
    \usepackage{parskip}
  }{% else
    \setlength{\parindent}{0pt}
    \setlength{\parskip}{6pt plus 2pt minus 1pt}}
}{% if KOMA class
  \KOMAoptions{parskip=half}}
\makeatother
\usepackage{xcolor}
\IfFileExists{xurl.sty}{\usepackage{xurl}}{} % add URL line breaks if available
\IfFileExists{bookmark.sty}{\usepackage{bookmark}}{\usepackage{hyperref}}
\hypersetup{
  pdftitle={IV PART 3},
  pdfauthor={Doron Zamir},
  hidelinks,
  pdfcreator={LaTeX via pandoc}}
\urlstyle{same} % disable monospaced font for URLs
\usepackage[margin=1in]{geometry}
\usepackage{color}
\usepackage{fancyvrb}
\newcommand{\VerbBar}{|}
\newcommand{\VERB}{\Verb[commandchars=\\\{\}]}
\DefineVerbatimEnvironment{Highlighting}{Verbatim}{commandchars=\\\{\}}
% Add ',fontsize=\small' for more characters per line
\usepackage{framed}
\definecolor{shadecolor}{RGB}{248,248,248}
\newenvironment{Shaded}{\begin{snugshade}}{\end{snugshade}}
\newcommand{\AlertTok}[1]{\textcolor[rgb]{0.94,0.16,0.16}{#1}}
\newcommand{\AnnotationTok}[1]{\textcolor[rgb]{0.56,0.35,0.01}{\textbf{\textit{#1}}}}
\newcommand{\AttributeTok}[1]{\textcolor[rgb]{0.77,0.63,0.00}{#1}}
\newcommand{\BaseNTok}[1]{\textcolor[rgb]{0.00,0.00,0.81}{#1}}
\newcommand{\BuiltInTok}[1]{#1}
\newcommand{\CharTok}[1]{\textcolor[rgb]{0.31,0.60,0.02}{#1}}
\newcommand{\CommentTok}[1]{\textcolor[rgb]{0.56,0.35,0.01}{\textit{#1}}}
\newcommand{\CommentVarTok}[1]{\textcolor[rgb]{0.56,0.35,0.01}{\textbf{\textit{#1}}}}
\newcommand{\ConstantTok}[1]{\textcolor[rgb]{0.00,0.00,0.00}{#1}}
\newcommand{\ControlFlowTok}[1]{\textcolor[rgb]{0.13,0.29,0.53}{\textbf{#1}}}
\newcommand{\DataTypeTok}[1]{\textcolor[rgb]{0.13,0.29,0.53}{#1}}
\newcommand{\DecValTok}[1]{\textcolor[rgb]{0.00,0.00,0.81}{#1}}
\newcommand{\DocumentationTok}[1]{\textcolor[rgb]{0.56,0.35,0.01}{\textbf{\textit{#1}}}}
\newcommand{\ErrorTok}[1]{\textcolor[rgb]{0.64,0.00,0.00}{\textbf{#1}}}
\newcommand{\ExtensionTok}[1]{#1}
\newcommand{\FloatTok}[1]{\textcolor[rgb]{0.00,0.00,0.81}{#1}}
\newcommand{\FunctionTok}[1]{\textcolor[rgb]{0.00,0.00,0.00}{#1}}
\newcommand{\ImportTok}[1]{#1}
\newcommand{\InformationTok}[1]{\textcolor[rgb]{0.56,0.35,0.01}{\textbf{\textit{#1}}}}
\newcommand{\KeywordTok}[1]{\textcolor[rgb]{0.13,0.29,0.53}{\textbf{#1}}}
\newcommand{\NormalTok}[1]{#1}
\newcommand{\OperatorTok}[1]{\textcolor[rgb]{0.81,0.36,0.00}{\textbf{#1}}}
\newcommand{\OtherTok}[1]{\textcolor[rgb]{0.56,0.35,0.01}{#1}}
\newcommand{\PreprocessorTok}[1]{\textcolor[rgb]{0.56,0.35,0.01}{\textit{#1}}}
\newcommand{\RegionMarkerTok}[1]{#1}
\newcommand{\SpecialCharTok}[1]{\textcolor[rgb]{0.00,0.00,0.00}{#1}}
\newcommand{\SpecialStringTok}[1]{\textcolor[rgb]{0.31,0.60,0.02}{#1}}
\newcommand{\StringTok}[1]{\textcolor[rgb]{0.31,0.60,0.02}{#1}}
\newcommand{\VariableTok}[1]{\textcolor[rgb]{0.00,0.00,0.00}{#1}}
\newcommand{\VerbatimStringTok}[1]{\textcolor[rgb]{0.31,0.60,0.02}{#1}}
\newcommand{\WarningTok}[1]{\textcolor[rgb]{0.56,0.35,0.01}{\textbf{\textit{#1}}}}
\usepackage{graphicx}
\makeatletter
\def\maxwidth{\ifdim\Gin@nat@width>\linewidth\linewidth\else\Gin@nat@width\fi}
\def\maxheight{\ifdim\Gin@nat@height>\textheight\textheight\else\Gin@nat@height\fi}
\makeatother
% Scale images if necessary, so that they will not overflow the page
% margins by default, and it is still possible to overwrite the defaults
% using explicit options in \includegraphics[width, height, ...]{}
\setkeys{Gin}{width=\maxwidth,height=\maxheight,keepaspectratio}
% Set default figure placement to htbp
\makeatletter
\def\fps@figure{htbp}
\makeatother
\setlength{\emergencystretch}{3em} % prevent overfull lines
\providecommand{\tightlist}{%
  \setlength{\itemsep}{0pt}\setlength{\parskip}{0pt}}
\setcounter{secnumdepth}{-\maxdimen} % remove section numbering
\ifluatex
  \usepackage{selnolig}  % disable illegal ligatures
\fi

\title{IV PART 3}
\author{Doron Zamir}
\date{5/9/2021}

\begin{document}
\maketitle

\hypertarget{getting-things-ready}{%
\section{Getting things ready}\label{getting-things-ready}}

\hypertarget{load-packages}{%
\subsection{Load Packages}\label{load-packages}}

\begin{Shaded}
\begin{Highlighting}[]
\ControlFlowTok{if}\NormalTok{ (}\SpecialCharTok{!}\FunctionTok{require}\NormalTok{(}\StringTok{"pacman"}\NormalTok{)) }\FunctionTok{install.packages}\NormalTok{(}\StringTok{"pacman"}\NormalTok{)}

\NormalTok{pacman}\SpecialCharTok{::}\FunctionTok{p\_load}\NormalTok{(}
\NormalTok{  tidyverse,}
\NormalTok{  vip,}
\NormalTok{  here,}
\NormalTok{  readxl,}
\NormalTok{  DataExplorer,}
\NormalTok{  GGally,}
\NormalTok{  np,}
\NormalTok{  ivtools}
\NormalTok{)}
\end{Highlighting}
\end{Shaded}

\hypertarget{load-data}{%
\subsection{Load Data}\label{load-data}}

Using data from Card (1993) that can be found
\href{https://davidcard.berkeley.edu/data_sets.html}{here}

\begin{Shaded}
\begin{Highlighting}[]
\NormalTok{schooling\_raw }\OtherTok{\textless{}{-}} \FunctionTok{read.table}\NormalTok{(}\StringTok{"Data/nls.dat"}\NormalTok{) }\SpecialCharTok{\%\textgreater{}\%}
  \FunctionTok{as\_data\_frame}\NormalTok{()}
\end{Highlighting}
\end{Shaded}

\hypertarget{change-variables-names}{%
\subsection{Change Variables Names}\label{change-variables-names}}

\hypertarget{tidy-up-the-data}{%
\subsection{Tidy up the data}\label{tidy-up-the-data}}

Selecting variables to work with, remove missing data, and create a new
dummy for collage proximity (regardless if it's a 4 year or 2 year
collage) for later analysis

\begin{Shaded}
\begin{Highlighting}[]
\NormalTok{schooling }\OtherTok{\textless{}{-}}\NormalTok{ schooling\_raw }\SpecialCharTok{\%\textgreater{}\%} 
  \FunctionTok{select}\NormalTok{(}
\NormalTok{    ed76,     }\CommentTok{\# Education {-} The treatment (t\_i)}
\NormalTok{    nearc4,   }\CommentTok{\# 4 year collage proximity {-} The IV (u\_i)  }
\NormalTok{    lwage78,  }\CommentTok{\# log wage in 78 {-} The output (y\_i)}
\NormalTok{    black,    }\CommentTok{\# Dummy for black {-} control (w\_i)}
\NormalTok{    age76     }\CommentTok{\# Age at 76       {-} control (w\_i) }
\NormalTok{    ) }\SpecialCharTok{\%\textgreater{}\%} 
  \FunctionTok{filter}\NormalTok{(lwage78 }\SpecialCharTok{!=} \StringTok{"."}\NormalTok{) }\SpecialCharTok{\%\textgreater{}\%}                      \CommentTok{\# remove missing values}
  \FunctionTok{mutate\_at}\NormalTok{(}\FunctionTok{vars}\NormalTok{(lwage78),}\FunctionTok{funs}\NormalTok{(as.numeric)) }\SpecialCharTok{\%\textgreater{}\%}    \CommentTok{\#make lwage numeric}
  \FunctionTok{mutate\_at}\NormalTok{(}\FunctionTok{vars}\NormalTok{(black,nearc4),}\FunctionTok{funs}\NormalTok{(as.factor)) }\CommentTok{\#set dummys as factor}
\end{Highlighting}
\end{Shaded}

\hypertarget{exploratory-data-analysis}{%
\subsection{Exploratory Data Analysis}\label{exploratory-data-analysis}}

using \texttt{DataExplorar}and \texttt{GGally} packages

\hypertarget{histograms}{%
\subsubsection{Histograms}\label{histograms}}

looking at histograms of \texttt{lwage78,\ ed76}

\begin{Shaded}
\begin{Highlighting}[]
\FunctionTok{plot\_histogram}\NormalTok{(schooling)}
\end{Highlighting}
\end{Shaded}

\includegraphics{ps2_iv_card_schholing_files/figure-latex/unnamed-chunk-3-1..svg}

It seems that there are a lot of observations with education less than
10, which might add noise to our analysis. we might need to consider
that later.

\hypertarget{correlations}{%
\subsubsection{Correlations:}\label{correlations}}

I used \texttt{ggparis()} from \texttt{GGally} package to plot every
pair of variables.

This is for the entire sample:

\begin{Shaded}
\begin{Highlighting}[]
\NormalTok{schooling }\SpecialCharTok{\%\textgreater{}\%}
  \FunctionTok{select}\NormalTok{(}\SpecialCharTok{{-}}\FunctionTok{c}\NormalTok{(age76)) }\SpecialCharTok{\%\textgreater{}\%}
  \FunctionTok{ggpairs}\NormalTok{()}
\end{Highlighting}
\end{Shaded}

\includegraphics{ps2_iv_card_schholing_files/figure-latex/unnamed-chunk-4-1..svg}

Since we seen that \texttt{ed76} has a long tail from the left, let's
see how's the correlation differ when trimming it

\begin{Shaded}
\begin{Highlighting}[]
\NormalTok{schooling }\SpecialCharTok{\%\textgreater{}\%}
  \FunctionTok{filter}\NormalTok{(ed76 }\SpecialCharTok{\textgreater{}=}\DecValTok{10}\NormalTok{ ) }\SpecialCharTok{\%\textgreater{}\%}
  \FunctionTok{select}\NormalTok{(}\SpecialCharTok{{-}}\FunctionTok{c}\NormalTok{(age76)) }\SpecialCharTok{\%\textgreater{}\%}
  \FunctionTok{ggpairs}\NormalTok{()}
\end{Highlighting}
\end{Shaded}

\includegraphics{ps2_iv_card_schholing_files/figure-latex/unnamed-chunk-5-1..svg}

Looking at the differences between the two plots, it seems that when
trimming for education \textgreater{} 10, we get a stronger correlation
between the instrument and the treatment. Then, trimming the data will
lead to better results:

\begin{Shaded}
\begin{Highlighting}[]
\NormalTok{schooling }\OtherTok{\textless{}{-}}\NormalTok{ schooling }\SpecialCharTok{\%\textgreater{}\%} 
  \FunctionTok{filter}\NormalTok{(ed76 }\SpecialCharTok{\textgreater{}=} \DecValTok{10}\NormalTok{)}
\end{Highlighting}
\end{Shaded}

\hypertarget{non-parametric-estimation}{%
\section{Non Parametric Estimation}\label{non-parametric-estimation}}

\hypertarget{general---boundaries-for-the-output}{%
\subsubsection{General - boundaries for the
output}\label{general---boundaries-for-the-output}}

Set \(K_0, K_1\) s.t \(\forall i, y_i (t) \in [K_0, K_1]\)

\begin{Shaded}
\begin{Highlighting}[]
\NormalTok{k\_0 }\OtherTok{\textless{}{-}} \FunctionTok{min}\NormalTok{(schooling}\SpecialCharTok{$}\NormalTok{lwage78)}
\NormalTok{k\_1 }\OtherTok{\textless{}{-}} \FunctionTok{max}\NormalTok{(schooling}\SpecialCharTok{$}\NormalTok{lwage78)}
\end{Highlighting}
\end{Shaded}

This will be useful for later calculations.

\hypertarget{iv-assumption}{%
\subsection{IV assumption}\label{iv-assumption}}

First, we assume that the connection between education and wage is the
same, regardless of collage proximity. More specially, we want to check
if \[E(lwage78|ed76, nearc4 =1) = E(lwage78|ed76, nearc4 =0)\]

let's check if this stands in the sample:

\begin{Shaded}
\begin{Highlighting}[]
\NormalTok{collage.labs }\OtherTok{\textless{}{-}} \FunctionTok{c}\NormalTok{(}\StringTok{"Far from Collage"}\NormalTok{, }\StringTok{"Close to Collage"}\NormalTok{)}
\FunctionTok{names}\NormalTok{(collage.labs) }\OtherTok{\textless{}{-}} \FunctionTok{c}\NormalTok{(}\DecValTok{0}\NormalTok{,}\DecValTok{1}\NormalTok{)}

\NormalTok{schooling }\SpecialCharTok{\%\textgreater{}\%}
  \FunctionTok{ggplot}\NormalTok{(}\FunctionTok{aes}\NormalTok{(ed76,lwage78)) }\SpecialCharTok{+}
    \FunctionTok{geom\_point}\NormalTok{() }\SpecialCharTok{+}
    \FunctionTok{geom\_smooth}\NormalTok{(}\AttributeTok{formula =}\NormalTok{ y }\SpecialCharTok{\textasciitilde{}}\NormalTok{ x) }\SpecialCharTok{+}
    \FunctionTok{facet\_wrap}\NormalTok{(}\StringTok{"nearc4"}\NormalTok{, }\AttributeTok{labeller =}\FunctionTok{labeller}\NormalTok{(}\AttributeTok{nearc4 =}\NormalTok{ collage.labs))}
\end{Highlighting}
\end{Shaded}

\includegraphics{ps2_iv_card_schholing_files/figure-latex/unnamed-chunk-8-1..svg}

seems like the assumption holds: the trend line for
(\texttt{lwage78\ \textasciitilde{}\ ed76}) looks the same in both
plots, i.e.~return for schooling doesn't seem to differ between those
who grew up close to 4 years collage and those who didn't.

\hypertarget{calculating-bounds}{%
\subsubsection{Calculating bounds}\label{calculating-bounds}}

I calculated bounds related to the IV assumption. That is, for every
treatment level \(t \in T\), I calculate bounds \(UB(t),LB(t)\) such
that:

\[
LB(t) \equiv \max_{u \in V} E[y|z=t, v= u] \cdot P(z=t|v=u) +K_0 \cdot P(z \ne t|v = u) \\
UB(t) \equiv \min_{u \in V} E[y|z=t, v= u] \cdot P(z=t|v=u) +K_1 \cdot P(z \ne t|v = u) 
\]

note that in this section, the treatment is the years of education,
i.e., \(t \in T = \{10,11,12...\}\)

\begin{Shaded}
\begin{Highlighting}[]
\NormalTok{iv\_bounds }\OtherTok{\textless{}{-}}\NormalTok{ schooling }\SpecialCharTok{\%\textgreater{}\%}
  \FunctionTok{select}\NormalTok{(}\SpecialCharTok{{-}}\FunctionTok{c}\NormalTok{(black, age76)) }\SpecialCharTok{\%\textgreater{}\%}     \CommentTok{\# Remove controls}
  \FunctionTok{group\_by}\NormalTok{(ed76, nearc4) }\SpecialCharTok{\%\textgreater{}\%}       \CommentTok{\# Grouping by treatment and variable {-} (t\_i,u\_i)}
  \FunctionTok{summarise}\NormalTok{(}
    \AttributeTok{exp =} \FunctionTok{mean}\NormalTok{(lwage78),           }\CommentTok{\# calculate expected value of Y for each group}
    \AttributeTok{count =} \FunctionTok{n}\NormalTok{()) }\SpecialCharTok{\%\textgreater{}\%}               \CommentTok{\# counting observation in each group}
  \FunctionTok{pivot\_wider}\NormalTok{(                     }\CommentTok{\# splitting count and exp vars by u\_i }
    \AttributeTok{values\_from =} \FunctionTok{c}\NormalTok{(exp,count),}
    \AttributeTok{names\_from =}\NormalTok{ nearc4,}
    \AttributeTok{values\_fill =} \DecValTok{0}\NormalTok{)}\SpecialCharTok{\%\textgreater{}\%}            \CommentTok{\# replacing missing data(no obs) with 0}
\NormalTok{    ungroup }\SpecialCharTok{\%\textgreater{}\%} 
  \FunctionTok{mutate}\NormalTok{(                          }\CommentTok{\# calculating probabilities}
    \AttributeTok{prob\_0 =}\NormalTok{ count\_0 }\SpecialCharTok{/} \FunctionTok{sum}\NormalTok{(count\_0),}
    \AttributeTok{prob\_1 =}\NormalTok{ count\_1 }\SpecialCharTok{/} \FunctionTok{sum}\NormalTok{(count\_1)}
\NormalTok{    ) }\SpecialCharTok{\%\textgreater{}\%}
    \FunctionTok{mutate}\NormalTok{(                                         }\CommentTok{\#calculating bounds}
      \AttributeTok{l\_bound\_0 =}\NormalTok{ prob\_0 }\SpecialCharTok{*}\NormalTok{ exp\_0 }\SpecialCharTok{+}\NormalTok{ k\_0 }\SpecialCharTok{*}\NormalTok{(}\DecValTok{1}\SpecialCharTok{{-}}\NormalTok{prob\_0), }\CommentTok{\#lower bound for u\_i = 0 }
      \AttributeTok{l\_bound\_1 =}\NormalTok{ prob\_1 }\SpecialCharTok{*}\NormalTok{ exp\_1 }\SpecialCharTok{+}\NormalTok{ k\_0 }\SpecialCharTok{*}\NormalTok{(}\DecValTok{1}\SpecialCharTok{{-}}\NormalTok{prob\_1), }\CommentTok{\#lower bound for u\_i = 1}
      \AttributeTok{u\_bound\_0 =}\NormalTok{ prob\_0 }\SpecialCharTok{*}\NormalTok{ exp\_0 }\SpecialCharTok{+}\NormalTok{ k\_1 }\SpecialCharTok{*}\NormalTok{(}\DecValTok{1}\SpecialCharTok{{-}}\NormalTok{prob\_0), }\CommentTok{\#upper bound for u\_i = 0}
      \AttributeTok{u\_bound\_1 =}\NormalTok{ prob\_1 }\SpecialCharTok{*}\NormalTok{ exp\_1 }\SpecialCharTok{+}\NormalTok{ k\_1 }\SpecialCharTok{*}\NormalTok{(}\DecValTok{1}\SpecialCharTok{{-}}\NormalTok{prob\_1)  }\CommentTok{\#upper bound for u\_i = 1}
\NormalTok{      ) }\SpecialCharTok{\%\textgreater{}\%}
    \FunctionTok{group\_by}\NormalTok{(ed76) }\SpecialCharTok{\%\textgreater{}\%}                              \CommentTok{\#grouping by ed76 level}
    \FunctionTok{mutate}\NormalTok{(}\AttributeTok{l\_bound =} \FunctionTok{max}\NormalTok{(l\_bound\_0,l\_bound\_1),      }\CommentTok{\#computing bounds}
           \AttributeTok{u\_bound =} \FunctionTok{min}\NormalTok{(u\_bound\_0,u\_bound\_1)) }\SpecialCharTok{\%\textgreater{}\%} 
    \FunctionTok{select}\NormalTok{(}\SpecialCharTok{{-}} \FunctionTok{contains}\NormalTok{(}\FunctionTok{c}\NormalTok{(}\StringTok{"\_0"}\NormalTok{,}\StringTok{"\_1"}\NormalTok{))) }\CommentTok{\# removing uneeded variables}
\end{Highlighting}
\end{Shaded}

this are the results for upper and lower bounds:

\begin{Shaded}
\begin{Highlighting}[]
\FunctionTok{print}\NormalTok{(iv\_bounds)}
\end{Highlighting}
\end{Shaded}

\begin{verbatim}
## # A tibble: 9 x 3
## # Groups:   ed76 [9]
##    ed76 l_bound u_bound
##   <int>   <dbl>   <dbl>
## 1    10    4.77    8.11
## 2    11    4.79    8.09
## 3    12    5.32    7.44
## 4    13    4.86    8.04
## 5    14    4.87    8.06
## 6    15    4.81    8.11
## 7    16    4.99    7.97
## 8    17    4.80    8.14
## 9    18    4.85    8.11
\end{verbatim}

\begin{Shaded}
\begin{Highlighting}[]
\FunctionTok{ggplot}\NormalTok{(iv\_bounds,}\FunctionTok{aes}\NormalTok{(}\AttributeTok{x =} \FunctionTok{as.factor}\NormalTok{(ed76), }\AttributeTok{color =} \FunctionTok{as.factor}\NormalTok{(ed76))) }\SpecialCharTok{+}
  \FunctionTok{geom\_errorbar}\NormalTok{(}\FunctionTok{aes}\NormalTok{(}\AttributeTok{ymin =}\NormalTok{ l\_bound,}\AttributeTok{ymax =}\NormalTok{ u\_bound, }\AttributeTok{width =} \FloatTok{0.3}\NormalTok{)) }\SpecialCharTok{+}
  \FunctionTok{labs}\NormalTok{( }\AttributeTok{x=} \StringTok{"ed76"}\NormalTok{, }\AttributeTok{y =}\StringTok{"E[lwage78|ed76]"}\NormalTok{ ) }\SpecialCharTok{+}
  \FunctionTok{theme}\NormalTok{(}\AttributeTok{legend.position =} \StringTok{"none"}\NormalTok{)}\SpecialCharTok{+}
   \FunctionTok{scale\_colour\_brewer}\NormalTok{(}\AttributeTok{palette =} \StringTok{"Set1"}\NormalTok{) }
\end{Highlighting}
\end{Shaded}

\includegraphics{ps2_iv_card_schholing_files/figure-latex/unnamed-chunk-11-1..svg}

we get quit big margins for every school of education. remembering that
the two picks of \texttt{ed76}'s distribution were in 12 and 16 years,
that what led into smaller margins for these levels of treatment.

\hypertarget{calculating-treatment-effect-te}{%
\subsubsection{Calculating treatment effect
(TE)}\label{calculating-treatment-effect-te}}

And for every \(t \in T\), meaning for every value of \texttt{ed76}, set
the treatment effect to be the difference between having treatment
\(t\), or \(t-1\).

\[
TE(t) = E[y(t) - y(t-1)] \\
\forall t > inf(T): \\
 LB(t) - UB(t-1) \le E[y(t) - y(t-1)] \le UB(t)- LB(t-1) 
\]

\begin{Shaded}
\begin{Highlighting}[]
\NormalTok{iv\_te }\OtherTok{\textless{}{-}}\NormalTok{ iv\_bounds[}\SpecialCharTok{{-}}\DecValTok{1}\NormalTok{,] }
\FunctionTok{colnames}\NormalTok{(iv\_te) }\OtherTok{\textless{}{-}} \FunctionTok{c}\NormalTok{(}\StringTok{"ed76"}\NormalTok{,}\StringTok{"min\_te"}\NormalTok{,}\StringTok{"max\_te"}\NormalTok{ )  }\CommentTok{\# calculating the diff acording to the formula above}
\ControlFlowTok{for}\NormalTok{ (i }\ControlFlowTok{in} \DecValTok{1}\SpecialCharTok{:}\FunctionTok{length}\NormalTok{(iv\_te}\SpecialCharTok{$}\NormalTok{ed76)) \{}
\NormalTok{  iv\_te}\SpecialCharTok{$}\NormalTok{min\_te[i] }\OtherTok{=}\NormalTok{ iv\_bounds}\SpecialCharTok{$}\NormalTok{l\_bound[i}\SpecialCharTok{+}\DecValTok{1}\NormalTok{] }\SpecialCharTok{{-}}\NormalTok{ iv\_bounds}\SpecialCharTok{$}\NormalTok{u\_bound[i]}
\NormalTok{  iv\_te}\SpecialCharTok{$}\NormalTok{max\_te[i] }\OtherTok{=}\NormalTok{ iv\_bounds}\SpecialCharTok{$}\NormalTok{u\_bound[i}\SpecialCharTok{+}\DecValTok{1}\NormalTok{] }\SpecialCharTok{{-}}\NormalTok{ iv\_bounds}\SpecialCharTok{$}\NormalTok{l\_bound[i]}
\NormalTok{\}}

\NormalTok{iv\_te}
\end{Highlighting}
\end{Shaded}

\begin{verbatim}
## # A tibble: 8 x 3
## # Groups:   ed76 [8]
##    ed76 min_te max_te
##   <int>  <dbl>  <dbl>
## 1    11  -3.31   3.32
## 2    12  -2.77   2.64
## 3    13  -2.57   2.72
## 4    14  -3.17   3.19
## 5    15  -3.24   3.25
## 6    16  -3.12   3.15
## 7    17  -3.17   3.15
## 8    18  -3.29   3.31
\end{verbatim}

\begin{Shaded}
\begin{Highlighting}[]
\NormalTok{iv\_te }\SpecialCharTok{\%\textgreater{}\%} \FunctionTok{ggplot}\NormalTok{(}\FunctionTok{aes}\NormalTok{(}
  \AttributeTok{x =} \FunctionTok{as.factor}\NormalTok{(ed76)}
\NormalTok{)) }\SpecialCharTok{+} \FunctionTok{geom\_errorbar}\NormalTok{(}\FunctionTok{aes}\NormalTok{(}
  \AttributeTok{ymin =}\NormalTok{ min\_te,}
  \AttributeTok{ymax =}\NormalTok{ max\_te}
\NormalTok{), }\AttributeTok{width=} \FloatTok{0.2}\NormalTok{, }\AttributeTok{color =} \StringTok{"red"}\NormalTok{) }\SpecialCharTok{+}
  \FunctionTok{labs}\NormalTok{(}\AttributeTok{x =} \StringTok{"ed76"}\NormalTok{, }\AttributeTok{y =} \StringTok{"TE(ed76)"}\NormalTok{) }\SpecialCharTok{+}
  \FunctionTok{geom\_hline}\NormalTok{(}\AttributeTok{yintercept=}\DecValTok{0}\NormalTok{, }\AttributeTok{linetype=}\StringTok{"dashed"}\NormalTok{)}
\end{Highlighting}
\end{Shaded}

\includegraphics{ps2_iv_card_schholing_files/figure-latex/unnamed-chunk-13-1..svg}

\hypertarget{discussion}{%
\subsubsection{Discussion}\label{discussion}}

The results does not seem to be useful: the margins for TE are very
wide, and also contains negative effects, which are the opposite of what
you would expect to get.

In the next part, I'll try to better define the treatment, and use MIV
assumption.

\hypertarget{miv-assumption}{%
\subsection{MIV assumption}\label{miv-assumption}}

\hypertarget{redefining-the-problem}{%
\subsubsection{Redefining the problem}\label{redefining-the-problem}}

In the previous section, I tried to non parameticaly estimate the
treatment effect of every year of extra education on wage, using
proximity to 4 year collage as an IV.

In this section, the output is still \texttt{lwage78}, but the treatment
is now set to be attendance to collage. more specifically. let
\(t \in \{0,1\}\) \[
t = \cases { 0 & ed76 < 12 \\ 1 & ed76 > 12}
\] So, we want to examine how going to collage affects wages, and we
want to do that using porximity to collage as an IV. We will not limit
proximity tocollage only to a 4 yrs collage, but rather to both 4 and 2
yrs.

\hypertarget{again-selecting-data}{%
\subsubsection{Again, selecting data}\label{again-selecting-data}}

\begin{Shaded}
\begin{Highlighting}[]
\NormalTok{schooling\_miv }\OtherTok{\textless{}{-}}\NormalTok{ schooling\_raw }\SpecialCharTok{\%\textgreater{}\%} 
  \FunctionTok{filter}\NormalTok{(lwage78 }\SpecialCharTok{!=} \StringTok{"."}\NormalTok{,ed76 }\SpecialCharTok{\textgreater{}=} \DecValTok{10}\NormalTok{) }\SpecialCharTok{\%\textgreater{}\%}  \CommentTok{\# remove missing values, only obs with 10 yrs or more of education}
  \FunctionTok{mutate}\NormalTok{(}
    \AttributeTok{nearc =} \FunctionTok{as.numeric}\NormalTok{(nearc4 }\SpecialCharTok{|}\NormalTok{ nearc2), }\CommentTok{\#new dummy for collage proximity}
    \AttributeTok{attendc =} \FunctionTok{as.numeric}\NormalTok{(ed76 }\SpecialCharTok{\textgreater{}} \DecValTok{12}\NormalTok{)      }\CommentTok{\#new dummy for collage attendence}
\NormalTok{  ) }\SpecialCharTok{\%\textgreater{}\%} 
  \FunctionTok{select}\NormalTok{(}
\NormalTok{    attendc,  }\CommentTok{\# attendenc to collage {-} The treatment (t\_i)}
\NormalTok{    nearc,    }\CommentTok{\#  collage proximity {-} The IV (u\_i)  }
\NormalTok{    lwage78,  }\CommentTok{\# log wage in 78 {-} The output (y\_i)}
\NormalTok{    black     }\CommentTok{\# Dummy for black {-} control (w\_i)}
\NormalTok{      ) }\SpecialCharTok{\%\textgreater{}\%} 
  \FunctionTok{mutate\_at}\NormalTok{(}\FunctionTok{vars}\NormalTok{(lwage78),}\FunctionTok{funs}\NormalTok{(as.numeric)) }\SpecialCharTok{\%\textgreater{}\%}    \CommentTok{\#make lwage numeric}
  \FunctionTok{mutate\_at}\NormalTok{(}\FunctionTok{vars}\NormalTok{(black,nearc,attendc),}\FunctionTok{funs}\NormalTok{(as.factor)) }\CommentTok{\#set dummys as factor}
\end{Highlighting}
\end{Shaded}

\hypertarget{the-miv-assumption}{%
\subsubsection{The MIV assumption}\label{the-miv-assumption}}

the MIV assumption, in our case, suggest that proximity to collage is
positively correlated with wages when controlling on education.
Specially:

\[
\forall t \in T, \\ E[lwage78|z=t, nearc = 1 ] \ge E[lwage78|z=t, nearc = 0 ]
\]

There are many reasons to assume that, specially the environment and
education might be better when closer to collage, due to academic staff
leaving nearby. But, it's interesting to see if this is true in our
sample:

\begin{Shaded}
\begin{Highlighting}[]
\NormalTok{attend.labs }\OtherTok{\textless{}{-}} \FunctionTok{c}\NormalTok{(}\StringTok{"Didn\textquotesingle{}t went to collage"}\NormalTok{, }\StringTok{"Went to collage"}\NormalTok{)}
\FunctionTok{names}\NormalTok{(attend.labs) }\OtherTok{\textless{}{-}} \FunctionTok{c}\NormalTok{(}\StringTok{"0"}\NormalTok{,}\StringTok{"1"}\NormalTok{)}

\NormalTok{schooling\_miv }\SpecialCharTok{\%\textgreater{}\%} 
  \FunctionTok{ggplot}\NormalTok{(}\FunctionTok{aes}\NormalTok{(lwage78)) }\SpecialCharTok{+}
  \FunctionTok{geom\_density}\NormalTok{()}\SpecialCharTok{+}
  \FunctionTok{facet\_grid}\NormalTok{(attendc}\SpecialCharTok{\textasciitilde{}}\NormalTok{nearc,}\AttributeTok{labeller =} \FunctionTok{labeller}\NormalTok{(}\AttributeTok{nearc =}\NormalTok{ collage.labs, }\AttributeTok{attendc =}\NormalTok{attend.labs))}
\end{Highlighting}
\end{Shaded}

\includegraphics{ps2_iv_card_schholing_files/figure-latex/unnamed-chunk-15-1..svg}

The difference is notable, but not very significant.

\hypertarget{miv-boundaries}{%
\subsubsection{MIV Boundaries}\label{miv-boundaries}}

I calculated bounds related to the MIV assumption. That is, for every
treatment level \(t \in T\) , I calculate bounds \(UB(t),LB(t)\) such
that:

\[ LB(t) \equiv \sum_{u \in V}[P(v = u)\cdot   \max_{u_1 \le u} \underline{b}(t,u_1)]\\
UB(t) \equiv \sum_{u \in V}[P(v = u)\cdot   \min_{u \le u_2}\bar{b}(t,u_2)]
\]

where

\[
\underline{b}(t,u) \equiv E[y|z=t, v= u] \cdot P(z=t|v=u) +K_0 \cdot P(z \ne t|v = u) \\
\bar{b}(t,u)\equiv E[y|z=t, v= u] \cdot P(z=t|v=u) +K_1 \cdot P(z \ne t|v = u)
\]

note that since \(u \in \{0,1\}\), and our assumption that

\[\forall t \in T, \\ E[lwage78|z=t, u = 1 ] \ge E[lwage78|z=t, u = 0 ]
\]

we get that:

\[ LB(t) \equiv \sum_{u \in V}[P(v = u)\cdot \max_{u_1 \le u} \underline{b}(t,u_1)] = P(u =0) \cdot \max_{u_1 \le 0} \underline{b}(t,u_1)+ P(u =1) \cdot \max_{u_1 \le 1} \underline{b}(t,u_1) \\
= P(u =0) \cdot \underline{b}(t,0)+ P(u =1) \cdot \max_{u_1 \le 1} \underline{b}(t,u_1)\\
UB(t) \equiv \sum_{u \in V}[P(v = u)\cdot   \min_{u \le u_2}\bar{b}(t,u_2)] =P(u =0) \cdot \min_{0 \le u_2} \bar{b}(t,u_2)+ P(u =1) \cdot \min_{1 \le u_2} \bar{b}(t,u_2) \\ P(u =0) \cdot \min_{0 \le u_2} \bar{b}(t,u_2)+ P(u =1) \cdot  \bar{b}(t,1)
\]

\begin{Shaded}
\begin{Highlighting}[]
\NormalTok{probs }\OtherTok{\textless{}{-}}\NormalTok{ schooling\_miv }\SpecialCharTok{\%\textgreater{}\%} 
  \FunctionTok{select}\NormalTok{(nearc) }\SpecialCharTok{\%\textgreater{}\%} \FunctionTok{group\_by}\NormalTok{(nearc) }\SpecialCharTok{\%\textgreater{}\%} 
  \FunctionTok{summarise}\NormalTok{(}\AttributeTok{count =} \FunctionTok{n}\NormalTok{()) }\SpecialCharTok{\%\textgreater{}\%} \FunctionTok{mutate}\NormalTok{(}
    \AttributeTok{prob =}\NormalTok{ count}\SpecialCharTok{/} \FunctionTok{sum}\NormalTok{(count)}
\NormalTok{  )}
\NormalTok{pr\_0 }\OtherTok{\textless{}{-}} \FunctionTok{as.numeric}\NormalTok{(probs[}\DecValTok{1}\NormalTok{,}\StringTok{"prob"}\NormalTok{])}
\NormalTok{pr\_1 }\OtherTok{\textless{}{-}} \DecValTok{1}\SpecialCharTok{{-}}\NormalTok{ pr\_0}

\NormalTok{MIV\_bounds\_all\_pop }\OtherTok{\textless{}{-}} 
\NormalTok{  schooling\_miv }\SpecialCharTok{\%\textgreater{}\%}
  \FunctionTok{select}\NormalTok{(}\SpecialCharTok{{-}}\FunctionTok{c}\NormalTok{(black)) }\SpecialCharTok{\%\textgreater{}\%}     \CommentTok{\# Remove controls}
  \FunctionTok{group\_by}\NormalTok{(attendc, nearc) }\SpecialCharTok{\%\textgreater{}\%}     \CommentTok{\# Grouping by treatment and variable {-} (t\_i,u\_i)}
  \FunctionTok{summarise}\NormalTok{(}
    \AttributeTok{exp =} \FunctionTok{mean}\NormalTok{(lwage78),           }\CommentTok{\# calculate expected value of Y for each group}
    \AttributeTok{count =} \FunctionTok{n}\NormalTok{()) }\SpecialCharTok{\%\textgreater{}\%}               \CommentTok{\# counting observation in each group}
  \FunctionTok{pivot\_wider}\NormalTok{(                     }\CommentTok{\# splitting count and exp vars by u\_i }
    \AttributeTok{values\_from =} \FunctionTok{c}\NormalTok{(exp,count),}
    \AttributeTok{names\_from =}\NormalTok{ nearc,}
    \AttributeTok{values\_fill =} \DecValTok{0}\NormalTok{)}\SpecialCharTok{\%\textgreater{}\%}            \CommentTok{\# replacing missing data(no obs) with 0}
\NormalTok{    ungroup }\SpecialCharTok{\%\textgreater{}\%} 
  \FunctionTok{mutate}\NormalTok{(                          }\CommentTok{\# calculating probabilities}
    \AttributeTok{prob\_0 =}\NormalTok{ count\_0 }\SpecialCharTok{/} \FunctionTok{sum}\NormalTok{(count\_0),}
    \AttributeTok{prob\_1 =}\NormalTok{ count\_1 }\SpecialCharTok{/} \FunctionTok{sum}\NormalTok{(count\_1)}
\NormalTok{    ) }\SpecialCharTok{\%\textgreater{}\%}
    \FunctionTok{mutate}\NormalTok{(                                         }\CommentTok{\#calculating bounds}
      \AttributeTok{l\_bound\_0 =}\NormalTok{ prob\_0 }\SpecialCharTok{*}\NormalTok{ exp\_0 }\SpecialCharTok{+}\NormalTok{ k\_0 }\SpecialCharTok{*}\NormalTok{(}\DecValTok{1}\SpecialCharTok{{-}}\NormalTok{prob\_0), }\CommentTok{\#lower bound for u\_i = 0 }
      \AttributeTok{l\_bound\_1 =}\NormalTok{ prob\_1 }\SpecialCharTok{*}\NormalTok{ exp\_1 }\SpecialCharTok{+}\NormalTok{ k\_0 }\SpecialCharTok{*}\NormalTok{(}\DecValTok{1}\SpecialCharTok{{-}}\NormalTok{prob\_1), }\CommentTok{\#lower bound for u\_i = 1}
      \AttributeTok{u\_bound\_0 =}\NormalTok{ prob\_0 }\SpecialCharTok{*}\NormalTok{ exp\_0 }\SpecialCharTok{+}\NormalTok{ k\_1 }\SpecialCharTok{*}\NormalTok{(}\DecValTok{1}\SpecialCharTok{{-}}\NormalTok{prob\_0), }\CommentTok{\#upper bound for u\_i = 0}
      \AttributeTok{u\_bound\_1 =}\NormalTok{ prob\_1 }\SpecialCharTok{*}\NormalTok{ exp\_1 }\SpecialCharTok{+}\NormalTok{ k\_1 }\SpecialCharTok{*}\NormalTok{(}\DecValTok{1}\SpecialCharTok{{-}}\NormalTok{prob\_1)  }\CommentTok{\#upper bound for u\_i = 1}
\NormalTok{      ) }\SpecialCharTok{\%\textgreater{}\%}
     \FunctionTok{group\_by}\NormalTok{(attendc) }\SpecialCharTok{\%\textgreater{}\%}             \CommentTok{\#grouping by attendence level}
     \FunctionTok{mutate}\NormalTok{(                           }\CommentTok{\#computing bounds}
       \AttributeTok{lb =}\NormalTok{ pr\_0 }\SpecialCharTok{*}\NormalTok{l\_bound\_0 }\SpecialCharTok{+}\NormalTok{pr\_1 }\SpecialCharTok{*} \FunctionTok{max}\NormalTok{(l\_bound\_0,l\_bound\_1),      }
       \AttributeTok{ub =}\NormalTok{ pr\_0 }\SpecialCharTok{*}\FunctionTok{min}\NormalTok{(u\_bound\_0,u\_bound\_1) }\SpecialCharTok{+}\NormalTok{ pr\_1 }\SpecialCharTok{*}\NormalTok{ u\_bound\_1}
\NormalTok{            ) }\SpecialCharTok{\%\textgreater{}\%} 
     \FunctionTok{select}\NormalTok{(}\SpecialCharTok{{-}} \FunctionTok{contains}\NormalTok{(}\FunctionTok{c}\NormalTok{(}\StringTok{"\_0"}\NormalTok{,}\StringTok{"\_1"}\NormalTok{))) }\SpecialCharTok{\%\textgreater{}\%}  \CommentTok{\# removing uneeded variables}
    \FunctionTok{pivot\_wider}\NormalTok{(}
      \AttributeTok{values\_from =}\NormalTok{ lb}\SpecialCharTok{:}\NormalTok{ub,}
      \AttributeTok{names\_from =}\NormalTok{ attendc}
\NormalTok{    ) }\SpecialCharTok{\%\textgreater{}\%} \FunctionTok{mutate}\NormalTok{(}\AttributeTok{pop =} \StringTok{"Entire Sample"}\NormalTok{)}
\end{Highlighting}
\end{Shaded}

For comparison, I will do the same calculation for sub populations:
black and white people.

let \(w \in W\) be a vector of covarients, in this example
\(W = \{black,white\}\), the calculation is the same but this time the
probabilities and expectations are conditional on \(w\).

\[ LB(t,w) \equiv \sum_{u \in V}[P(v = u|w)\cdot   \max_{u_1 \le u} \underline{b}(t,u_1,w)]\\
UB(t,w) \equiv \sum_{u \in V}[P(v = u|w)\cdot   \min_{u \le u_2}\bar{b}(t,u_2,w)]
\]

\begin{Shaded}
\begin{Highlighting}[]
\NormalTok{miv\_bounds }\OtherTok{\textless{}{-}}\NormalTok{ MIV\_bounds\_all\_pop }\SpecialCharTok{\%\textgreater{}\%}  \CommentTok{\# this df is for storing the bounds}
  \FunctionTok{bind\_rows}\NormalTok{(MIV\_bounds\_only\_blacks,MIV\_bounds\_only\_whites)}

\NormalTok{miv\_bounds\_plot }\OtherTok{\textless{}{-}}\NormalTok{ miv\_bounds }\SpecialCharTok{\%\textgreater{}\%}   \CommentTok{\# this df is for storing the bounds, in a way good for plotting}
  \FunctionTok{pivot\_longer}\NormalTok{(}\AttributeTok{cols =} \SpecialCharTok{!}\StringTok{"pop"}\NormalTok{,}
               \AttributeTok{names\_to =} \FunctionTok{c}\NormalTok{(}\StringTok{".value"}\NormalTok{,}\StringTok{"treatment"}\NormalTok{),}
               \AttributeTok{names\_pattern =} \StringTok{"(.*)\_(.)"}\NormalTok{) }\SpecialCharTok{\%\textgreater{}\%}  
  \FunctionTok{mutate}\NormalTok{(}\AttributeTok{treatment =} \FunctionTok{factor}\NormalTok{(treatment,}\AttributeTok{levels=}\FunctionTok{c}\NormalTok{(}\DecValTok{0}\NormalTok{,}\DecValTok{1}\NormalTok{), }\AttributeTok{labels=}\FunctionTok{c}\NormalTok{(}\StringTok{"Didn\textquotesingle{}t attend collage"}\NormalTok{, }\StringTok{"Attend collage"}\NormalTok{)))}
         
\NormalTok{miv\_bounds\_plot }\SpecialCharTok{\%\textgreater{}\%} \FunctionTok{ggplot}\NormalTok{(}\FunctionTok{aes}\NormalTok{(}
  \AttributeTok{x =}\NormalTok{ pop)) }\SpecialCharTok{+}
  \FunctionTok{geom\_errorbar}\NormalTok{(}\AttributeTok{width =} \FloatTok{0.2}\NormalTok{, }\FunctionTok{aes}\NormalTok{(}
    \AttributeTok{ymin =}\NormalTok{ lb,}
    \AttributeTok{ymax =}\NormalTok{ ub,}
    \AttributeTok{colour =}\NormalTok{ treatment),}
    \AttributeTok{position =} \StringTok{"dodge"}\NormalTok{) }\SpecialCharTok{+}
  \FunctionTok{labs}\NormalTok{(}\AttributeTok{x =} \StringTok{"Subpopulation"}\NormalTok{, }\AttributeTok{y =} \StringTok{"E[lwage|attenc]"}\NormalTok{, }\AttributeTok{color =} \StringTok{"Treatment"}\NormalTok{)}
\end{Highlighting}
\end{Shaded}

\includegraphics{ps2_iv_card_schholing_files/figure-latex/unnamed-chunk-19-1..svg}

\begin{Shaded}
\begin{Highlighting}[]
\NormalTok{miv\_te  }\OtherTok{\textless{}{-}}\NormalTok{  miv\_bounds }\SpecialCharTok{\%\textgreater{}\%} \FunctionTok{group\_by}\NormalTok{(pop) }\SpecialCharTok{\%\textgreater{}\%}  \CommentTok{\#calculating treatmnet effects}
  \FunctionTok{summarise}\NormalTok{(}\AttributeTok{min\_te =}\NormalTok{ ub\_0 }\SpecialCharTok{{-}}\NormalTok{ lb\_1,}
            \AttributeTok{max\_te =}\NormalTok{ ub\_1 }\SpecialCharTok{{-}}\NormalTok{ lb\_0)}

\NormalTok{miv\_te }\SpecialCharTok{\%\textgreater{}\%} \FunctionTok{ggplot}\NormalTok{(}\FunctionTok{aes}\NormalTok{(}\AttributeTok{x =}\NormalTok{ pop)) }\SpecialCharTok{+}
  \FunctionTok{geom\_errorbar}\NormalTok{(}\AttributeTok{width =} \FloatTok{0.2}\NormalTok{, }\FunctionTok{aes}\NormalTok{(}
    \AttributeTok{ymin =}\NormalTok{ min\_te,}
    \AttributeTok{ymax =}\NormalTok{ max\_te),}
    \AttributeTok{color =} \StringTok{"red"}\NormalTok{) }\SpecialCharTok{+}
  \FunctionTok{labs}\NormalTok{( }\AttributeTok{x=} \StringTok{"Population"}\NormalTok{, }\AttributeTok{y =} \StringTok{"TE(t|w)"}\NormalTok{)}
\end{Highlighting}
\end{Shaded}

\includegraphics{ps2_iv_card_schholing_files/figure-latex/unnamed-chunk-20-1..svg}

\hypertarget{discussion-1}{%
\subsubsection{Discussion}\label{discussion-1}}

We obtained nonparaetric bounds on the ATE for two sub populations.

We can see that the return for going to collage is strictly positive for
all subgroups, and that the segment in which the TE lies is bigger for
blacks then for whites. This corresponds to the assumption that the
process determining wages is sensitive to race (throw discrimination,
profession selection. etc.) as well as to education.

\end{document}
